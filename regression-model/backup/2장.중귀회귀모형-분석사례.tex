% Options for packages loaded elsewhere
\PassOptionsToPackage{unicode}{hyperref}
\PassOptionsToPackage{hyphens}{url}
%
\documentclass[
]{article}
\usepackage{amsmath,amssymb}
\usepackage{lmodern}
\usepackage{iftex}
\ifPDFTeX
  \usepackage[T1]{fontenc}
  \usepackage[utf8]{inputenc}
  \usepackage{textcomp} % provide euro and other symbols
\else % if luatex or xetex
  \usepackage{unicode-math}
  \defaultfontfeatures{Scale=MatchLowercase}
  \defaultfontfeatures[\rmfamily]{Ligatures=TeX,Scale=1}
\fi
% Use upquote if available, for straight quotes in verbatim environments
\IfFileExists{upquote.sty}{\usepackage{upquote}}{}
\IfFileExists{microtype.sty}{% use microtype if available
  \usepackage[]{microtype}
  \UseMicrotypeSet[protrusion]{basicmath} % disable protrusion for tt fonts
}{}
\makeatletter
\@ifundefined{KOMAClassName}{% if non-KOMA class
  \IfFileExists{parskip.sty}{%
    \usepackage{parskip}
  }{% else
    \setlength{\parindent}{0pt}
    \setlength{\parskip}{6pt plus 2pt minus 1pt}}
}{% if KOMA class
  \KOMAoptions{parskip=half}}
\makeatother
\usepackage{xcolor}
\usepackage[margin=1in]{geometry}
\usepackage{color}
\usepackage{fancyvrb}
\newcommand{\VerbBar}{|}
\newcommand{\VERB}{\Verb[commandchars=\\\{\}]}
\DefineVerbatimEnvironment{Highlighting}{Verbatim}{commandchars=\\\{\}}
% Add ',fontsize=\small' for more characters per line
\usepackage{framed}
\definecolor{shadecolor}{RGB}{248,248,248}
\newenvironment{Shaded}{\begin{snugshade}}{\end{snugshade}}
\newcommand{\AlertTok}[1]{\textcolor[rgb]{0.94,0.16,0.16}{#1}}
\newcommand{\AnnotationTok}[1]{\textcolor[rgb]{0.56,0.35,0.01}{\textbf{\textit{#1}}}}
\newcommand{\AttributeTok}[1]{\textcolor[rgb]{0.77,0.63,0.00}{#1}}
\newcommand{\BaseNTok}[1]{\textcolor[rgb]{0.00,0.00,0.81}{#1}}
\newcommand{\BuiltInTok}[1]{#1}
\newcommand{\CharTok}[1]{\textcolor[rgb]{0.31,0.60,0.02}{#1}}
\newcommand{\CommentTok}[1]{\textcolor[rgb]{0.56,0.35,0.01}{\textit{#1}}}
\newcommand{\CommentVarTok}[1]{\textcolor[rgb]{0.56,0.35,0.01}{\textbf{\textit{#1}}}}
\newcommand{\ConstantTok}[1]{\textcolor[rgb]{0.00,0.00,0.00}{#1}}
\newcommand{\ControlFlowTok}[1]{\textcolor[rgb]{0.13,0.29,0.53}{\textbf{#1}}}
\newcommand{\DataTypeTok}[1]{\textcolor[rgb]{0.13,0.29,0.53}{#1}}
\newcommand{\DecValTok}[1]{\textcolor[rgb]{0.00,0.00,0.81}{#1}}
\newcommand{\DocumentationTok}[1]{\textcolor[rgb]{0.56,0.35,0.01}{\textbf{\textit{#1}}}}
\newcommand{\ErrorTok}[1]{\textcolor[rgb]{0.64,0.00,0.00}{\textbf{#1}}}
\newcommand{\ExtensionTok}[1]{#1}
\newcommand{\FloatTok}[1]{\textcolor[rgb]{0.00,0.00,0.81}{#1}}
\newcommand{\FunctionTok}[1]{\textcolor[rgb]{0.00,0.00,0.00}{#1}}
\newcommand{\ImportTok}[1]{#1}
\newcommand{\InformationTok}[1]{\textcolor[rgb]{0.56,0.35,0.01}{\textbf{\textit{#1}}}}
\newcommand{\KeywordTok}[1]{\textcolor[rgb]{0.13,0.29,0.53}{\textbf{#1}}}
\newcommand{\NormalTok}[1]{#1}
\newcommand{\OperatorTok}[1]{\textcolor[rgb]{0.81,0.36,0.00}{\textbf{#1}}}
\newcommand{\OtherTok}[1]{\textcolor[rgb]{0.56,0.35,0.01}{#1}}
\newcommand{\PreprocessorTok}[1]{\textcolor[rgb]{0.56,0.35,0.01}{\textit{#1}}}
\newcommand{\RegionMarkerTok}[1]{#1}
\newcommand{\SpecialCharTok}[1]{\textcolor[rgb]{0.00,0.00,0.00}{#1}}
\newcommand{\SpecialStringTok}[1]{\textcolor[rgb]{0.31,0.60,0.02}{#1}}
\newcommand{\StringTok}[1]{\textcolor[rgb]{0.31,0.60,0.02}{#1}}
\newcommand{\VariableTok}[1]{\textcolor[rgb]{0.00,0.00,0.00}{#1}}
\newcommand{\VerbatimStringTok}[1]{\textcolor[rgb]{0.31,0.60,0.02}{#1}}
\newcommand{\WarningTok}[1]{\textcolor[rgb]{0.56,0.35,0.01}{\textbf{\textit{#1}}}}
\usepackage{graphicx}
\makeatletter
\def\maxwidth{\ifdim\Gin@nat@width>\linewidth\linewidth\else\Gin@nat@width\fi}
\def\maxheight{\ifdim\Gin@nat@height>\textheight\textheight\else\Gin@nat@height\fi}
\makeatother
% Scale images if necessary, so that they will not overflow the page
% margins by default, and it is still possible to overwrite the defaults
% using explicit options in \includegraphics[width, height, ...]{}
\setkeys{Gin}{width=\maxwidth,height=\maxheight,keepaspectratio}
% Set default figure placement to htbp
\makeatletter
\def\fps@figure{htbp}
\makeatother
\setlength{\emergencystretch}{3em} % prevent overfull lines
\providecommand{\tightlist}{%
  \setlength{\itemsep}{0pt}\setlength{\parskip}{0pt}}
\setcounter{secnumdepth}{-\maxdimen} % remove section numbering
\ifLuaTeX
  \usepackage{selnolig}  % disable illegal ligatures
\fi
\IfFileExists{bookmark.sty}{\usepackage{bookmark}}{\usepackage{hyperref}}
\IfFileExists{xurl.sty}{\usepackage{xurl}}{} % add URL line breaks if available
\urlstyle{same} % disable monospaced font for URLs
\hypersetup{
  pdftitle={regression model chapter 2},
  pdfauthor={robinhwp},
  hidelinks,
  pdfcreator={LaTeX via pandoc}}

\title{regression model chapter 2}
\author{robinhwp}
\date{2023-03-14}

\begin{document}
\maketitle

\hypertarget{uxc81c2uxc7a5-uxbd84uxc11duxc0acuxb8405uxac15}{%
\section{제2장
분석사례(5강)}\label{uxc81c2uxc7a5-uxbd84uxc11duxc0acuxb8405uxac15}}

\hypertarget{uxc790uxb8ccuxc77duxae30}{%
\subsection{1) 자료읽기}\label{uxc790uxb8ccuxc77duxae30}}

\begin{Shaded}
\begin{Highlighting}[]
\FunctionTok{library}\NormalTok{(xlsx) }
\NormalTok{chemical}\OtherTok{=}\FunctionTok{read.xlsx}\NormalTok{(}\StringTok{"./data/chemical.xlsx"}\NormalTok{, }\DecValTok{1}\NormalTok{)}
\FunctionTok{head}\NormalTok{(chemical)}
\end{Highlighting}
\end{Shaded}

\begin{verbatim}
##   id speed temp loss
## 1  1    70   20   15
## 2  2    80   27   42
## 3  3    75   25   37
## 4  4    62   24   28
## 5  5    65   23   20
## 6  6    58   18   14
\end{verbatim}

\hypertarget{uxae30uxc220uxd1b5uxacc4uxb7c9-uxbc0f-uxc0c1uxad00uxacc4uxc218-uxbcf4uxae30}{%
\subsection{2) 기술통계량 및 상관계수
보기}\label{uxae30uxc220uxd1b5uxacc4uxb7c9-uxbc0f-uxc0c1uxad00uxacc4uxc218-uxbcf4uxae30}}

\begin{Shaded}
\begin{Highlighting}[]
\CommentTok{\# 자료 요약}
\FunctionTok{summary}\NormalTok{(chemical[,}\SpecialCharTok{{-}}\DecValTok{1}\NormalTok{]) }
\end{Highlighting}
\end{Shaded}

\begin{verbatim}
##      speed           temp            loss      
##  Min.   :50.0   Min.   :17.00   Min.   : 8.00  
##  1st Qu.:57.5   1st Qu.:19.50   1st Qu.:13.75  
##  Median :61.0   Median :20.50   Median :16.50  
##  Mean   :63.0   Mean   :21.50   Mean   :20.25  
##  3rd Qu.:70.5   3rd Qu.:24.25   3rd Qu.:24.25  
##  Max.   :80.0   Max.   :27.00   Max.   :42.00
\end{verbatim}

\begin{Shaded}
\begin{Highlighting}[]
\CommentTok{\# 상관계수}
\FunctionTok{cor}\NormalTok{(chemical[,}\SpecialCharTok{{-}}\DecValTok{1}\NormalTok{])}
\end{Highlighting}
\end{Shaded}

\begin{verbatim}
##           speed      temp      loss
## speed 1.0000000 0.8023847 0.8548423
## temp  0.8023847 1.0000000 0.8953498
## loss  0.8548423 0.8953498 1.0000000
\end{verbatim}

\hypertarget{uxc0b0uxc810uxb3c4-uxadf8uxb9acuxae30}{%
\subsection{3) 산점도
그리기}\label{uxc0b0uxc810uxb3c4-uxadf8uxb9acuxae30}}

\begin{Shaded}
\begin{Highlighting}[]
\FunctionTok{par}\NormalTok{(}\AttributeTok{mfrow=}\FunctionTok{c}\NormalTok{(}\DecValTok{1}\NormalTok{,}\DecValTok{2}\NormalTok{), }\AttributeTok{pty=}\StringTok{"s"}\NormalTok{) }
\FunctionTok{plot}\NormalTok{(chemical}\SpecialCharTok{$}\NormalTok{speed, chemical}\SpecialCharTok{$}\NormalTok{loss, }\AttributeTok{pch=}\DecValTok{19}\NormalTok{)}
\FunctionTok{plot}\NormalTok{(chemical}\SpecialCharTok{$}\NormalTok{temp, chemical}\SpecialCharTok{$}\NormalTok{loss, }\AttributeTok{pch=}\DecValTok{19}\NormalTok{)}
\end{Highlighting}
\end{Shaded}

\includegraphics{2장.중귀회귀모형-분석사례_files/figure-latex/unnamed-chunk-3-1.pdf}

\hypertarget{uxd68cuxadc0uxbaa8uxd615-uxc801uxd569uxd558uxae30}{%
\subsection{4) 회귀모형
적합하기}\label{uxd68cuxadc0uxbaa8uxd615-uxc801uxd569uxd558uxae30}}

\begin{Shaded}
\begin{Highlighting}[]
\NormalTok{chemical.lm }\OtherTok{=} \FunctionTok{lm}\NormalTok{(loss }\SpecialCharTok{\textasciitilde{}}\NormalTok{ speed}\SpecialCharTok{+}\NormalTok{temp, }\AttributeTok{data=}\NormalTok{chemical) }
\FunctionTok{summary}\NormalTok{(chemical.lm)}
\end{Highlighting}
\end{Shaded}

\begin{verbatim}
## 
## Call:
## lm(formula = loss ~ speed + temp, data = chemical)
## 
## Residuals:
##     Min      1Q  Median      3Q     Max 
## -7.7699 -2.4093  0.2795  3.4019  4.9654 
## 
## Coefficients:
##             Estimate Std. Error t value Pr(>|t|)    
## (Intercept) -47.6243     9.4580  -5.035 0.000704 ***
## speed         0.4216     0.2350   1.794 0.106360    
## temp          1.9217     0.6977   2.754 0.022316 *  
## ---
## Signif. codes:  0 '***' 0.001 '**' 0.01 '*' 0.05 '.' 0.1 ' ' 1
## 
## Residual standard error: 4.465 on 9 degrees of freedom
## Multiple R-squared:  0.8539, Adjusted R-squared:  0.8214 
## F-statistic:  26.3 on 2 and 9 DF,  p-value: 0.0001741
\end{verbatim}

\hypertarget{uxcd94uxc815uxb41c-uxd68cuxadc0uxbc29uxc815uxc2dd-1}{%
\subsubsection{추정된 회귀방정식
1)}\label{uxcd94uxc815uxb41c-uxd68cuxadc0uxbc29uxc815uxc2dd-1}}

\begin{Shaded}
\begin{Highlighting}[]
\CommentTok{\# 추정된 회귀방정식 1) (chemical.lm$model과 coef(chemical.lm) 함수를 이용)}
\NormalTok{str\_l }\OtherTok{=} \FunctionTok{paste0}\NormalTok{(}\StringTok{"hat\_"}\NormalTok{, }\FunctionTok{names}\NormalTok{(chemical.lm}\SpecialCharTok{$}\NormalTok{model)[}\DecValTok{1}\NormalTok{])}
\NormalTok{str\_r }\OtherTok{=} \FunctionTok{paste0}\NormalTok{( }\FunctionTok{round}\NormalTok{(}\FunctionTok{coef}\NormalTok{(chemical.lm)[}\StringTok{"(Intercept)"}\NormalTok{], }\DecValTok{3}\NormalTok{))}
\ControlFlowTok{for}\NormalTok{(i }\ControlFlowTok{in} \DecValTok{2}\SpecialCharTok{:}\FunctionTok{length}\NormalTok{(}\FunctionTok{coef}\NormalTok{(chemical.lm))) }
\NormalTok{\{}
\NormalTok{  str\_r }\OtherTok{=} \FunctionTok{paste0}\NormalTok{( str\_r, }\StringTok{" + "}\NormalTok{, }\FunctionTok{round}\NormalTok{(}\FunctionTok{coef}\NormalTok{(chemical.lm)[}\FunctionTok{names}\NormalTok{(chemical.lm}\SpecialCharTok{$}\NormalTok{model)[i]], }\DecValTok{3}\NormalTok{),}
                  \StringTok{"*"}\NormalTok{, }\FunctionTok{names}\NormalTok{(chemical.lm}\SpecialCharTok{$}\NormalTok{model)[i])}
\NormalTok{\}}
\FunctionTok{paste0}\NormalTok{(}\StringTok{"추정된 회귀방정식: "}\NormalTok{, str\_l, }\StringTok{" = "}\NormalTok{, str\_r)}
\end{Highlighting}
\end{Shaded}

\begin{verbatim}
## [1] "추정된 회귀방정식: hat_loss = -47.624 + 0.422*speed + 1.922*temp"
\end{verbatim}

\hypertarget{uxcd94uxc815uxb41c-uxd68cuxadc0uxbc29uxc815uxc2dd-2}{%
\subsubsection{추정된 회귀방정식
2)}\label{uxcd94uxc815uxb41c-uxd68cuxadc0uxbc29uxc815uxc2dd-2}}

\begin{Shaded}
\begin{Highlighting}[]
\CommentTok{\# 추정된 회귀방정식 2) 최소제곱법으로 β를 행렬방정식으로 계산}
\NormalTok{mX}\OtherTok{=}\FunctionTok{as.matrix}\NormalTok{(}\FunctionTok{cbind}\NormalTok{(}\DecValTok{1}\NormalTok{, chemical}\SpecialCharTok{$}\NormalTok{speed, chemical}\SpecialCharTok{$}\NormalTok{temp))}
\NormalTok{mY}\OtherTok{=}\FunctionTok{as.matrix}\NormalTok{(chemical}\SpecialCharTok{$}\NormalTok{loss)}
\NormalTok{mβ}\OtherTok{=}\FunctionTok{solve}\NormalTok{(}\FunctionTok{t}\NormalTok{(mX)}\SpecialCharTok{\%*\%}\NormalTok{mX)}\SpecialCharTok{\%*\%}\FunctionTok{t}\NormalTok{(mX)}\SpecialCharTok{\%*\%}\NormalTok{mY}
\FunctionTok{paste0}\NormalTok{(}\StringTok{" 최소제곱법 이용 : hat\_loss = "}\NormalTok{, }\FunctionTok{round}\NormalTok{(mβ, }\DecValTok{3}\NormalTok{)[}\DecValTok{1}\NormalTok{], }\StringTok{" + "}\NormalTok{,}
       \FunctionTok{round}\NormalTok{(mβ, }\DecValTok{3}\NormalTok{)[}\DecValTok{2}\NormalTok{], }\StringTok{"*speed + "}\NormalTok{,}
       \FunctionTok{round}\NormalTok{(mβ, }\DecValTok{3}\NormalTok{)[}\DecValTok{3}\NormalTok{], }\StringTok{"*temp"}\NormalTok{)}
\end{Highlighting}
\end{Shaded}

\begin{verbatim}
## [1] " 최소제곱법 이용 : hat_loss = -47.624 + 0.422*speed + 1.922*temp"
\end{verbatim}

\hypertarget{uxacb0uxc815uxacc4uxc218}{%
\subsubsection{결정계수}\label{uxacb0uxc815uxacc4uxc218}}

\begin{Shaded}
\begin{Highlighting}[]
\CommentTok{\# 결정계수}
\NormalTok{chemical.lm.summary}\OtherTok{=}\FunctionTok{summary}\NormalTok{(chemical.lm)}
\FunctionTok{paste0}\NormalTok{(}\StringTok{"결정계수 R.squared = "}\NormalTok{, }\FunctionTok{round}\NormalTok{(chemical.lm.summary}\SpecialCharTok{$}\NormalTok{r.squared, }\DecValTok{3}\NormalTok{), }\StringTok{"으로 "}\NormalTok{,}
       \FunctionTok{round}\NormalTok{(chemical.lm.summary}\SpecialCharTok{$}\NormalTok{r.squared}\SpecialCharTok{*}\DecValTok{100}\NormalTok{,}\DecValTok{1}\NormalTok{), }\StringTok{"\% 설설명력이 있다."}\NormalTok{)}
\end{Highlighting}
\end{Shaded}

\begin{verbatim}
## [1] "결정계수 R.squared = 0.854으로 85.4% 설설명력이 있다."
\end{verbatim}

\hypertarget{p-value}{%
\subsubsection{p-value}\label{p-value}}

\begin{Shaded}
\begin{Highlighting}[]
\CommentTok{\# p{-}value}
\NormalTok{α }\OtherTok{=} \FloatTok{0.05}
\ControlFlowTok{for}\NormalTok{(i }\ControlFlowTok{in} \DecValTok{2}\SpecialCharTok{:}\FunctionTok{length}\NormalTok{(}\FunctionTok{coef}\NormalTok{(chemical.lm))) }
\NormalTok{\{}
\NormalTok{  p.value }\OtherTok{=} \FunctionTok{round}\NormalTok{(chemical.lm.summary}\SpecialCharTok{$}\NormalTok{coefficients[}\FunctionTok{names}\NormalTok{(chemical.lm}\SpecialCharTok{$}\NormalTok{model)[i], }\StringTok{"Pr(\textgreater{}|t|)"}\NormalTok{], }\DecValTok{5}\NormalTok{)}
  \ControlFlowTok{if}\NormalTok{ (p.value }\SpecialCharTok{\textless{}}\NormalTok{ α )}
\NormalTok{  \{}
    \FunctionTok{print}\NormalTok{(}\FunctionTok{paste0}\NormalTok{( }\FunctionTok{names}\NormalTok{(chemical.lm}\SpecialCharTok{$}\NormalTok{model)[i], }\StringTok{"의 p{-}value 가 "}\NormalTok{, p.value, }\StringTok{"으로서 "}\NormalTok{, }
                  \FunctionTok{names}\NormalTok{(chemical.lm}\SpecialCharTok{$}\NormalTok{model)[}\DecValTok{1}\NormalTok{], }\StringTok{"를(을) 설명하는데 유의하다"}\NormalTok{))}
\NormalTok{  \}}
  \ControlFlowTok{else}
\NormalTok{  \{}
    \FunctionTok{print}\NormalTok{(}\FunctionTok{paste0}\NormalTok{( }\FunctionTok{names}\NormalTok{(chemical.lm}\SpecialCharTok{$}\NormalTok{model)[i], }\StringTok{"의 p{-}value 가 "}\NormalTok{, p.value, }\StringTok{"으로서 "}\NormalTok{, }
                  \FunctionTok{names}\NormalTok{(chemical.lm}\SpecialCharTok{$}\NormalTok{model)[}\DecValTok{1}\NormalTok{], }\StringTok{"를(을) 설명하는데 그리 큰 영향을 준다고 할 수 없다."}\NormalTok{))}
\NormalTok{  \}}
\NormalTok{\}}
\end{Highlighting}
\end{Shaded}

\begin{verbatim}
## [1] "speed의 p-value 가 0.10636으로서 loss를(을) 설명하는데 그리 큰 영향을 준다고 할 수 없다."
## [1] "temp의 p-value 가 0.02232으로서 loss를(을) 설명하는데 유의하다"
\end{verbatim}

\hypertarget{uxbd84uxc0b0uxbd84uxc11duxd45c-uxad6cuxd558uxae30}{%
\subsection{5) 분산분석표
구하기}\label{uxbd84uxc0b0uxbd84uxc11duxd45c-uxad6cuxd558uxae30}}

\begin{Shaded}
\begin{Highlighting}[]
\CommentTok{\# 4) 분산분석표 구하기}
\NormalTok{chemical.lm.anova}\OtherTok{=}\FunctionTok{anova}\NormalTok{(chemical.lm)}
\NormalTok{AVT }\OtherTok{=} \FunctionTok{matrix}\NormalTok{(}\FunctionTok{c}\NormalTok{(}\DecValTok{0}\NormalTok{, }\DecValTok{0}\NormalTok{, }\DecValTok{0}\NormalTok{, }\DecValTok{0}\NormalTok{, }\DecValTok{0}\NormalTok{, }\DecValTok{0}\NormalTok{, }\DecValTok{0}\NormalTok{, }\DecValTok{0}\NormalTok{, }\DecValTok{0}\NormalTok{, }\DecValTok{0}\NormalTok{, }\DecValTok{0}\NormalTok{, }\DecValTok{0}\NormalTok{, }\DecValTok{0}\NormalTok{, }\DecValTok{0}\NormalTok{, }\DecValTok{0}\NormalTok{), }\AttributeTok{nrow=}\DecValTok{3}\NormalTok{)}
\FunctionTok{colnames}\NormalTok{(AVT)}\OtherTok{=}\FunctionTok{colnames}\NormalTok{(chemical.lm.anova)}
\FunctionTok{rownames}\NormalTok{(AVT)}\OtherTok{=}\FunctionTok{c}\NormalTok{(}\StringTok{"Regression"}\NormalTok{, }\StringTok{"Residuals"}\NormalTok{, }\StringTok{"Total"}\NormalTok{)}

\ControlFlowTok{for}\NormalTok{(i }\ControlFlowTok{in} \DecValTok{1}\SpecialCharTok{:}\NormalTok{(}\FunctionTok{length}\NormalTok{(}\FunctionTok{rownames}\NormalTok{(AVT))}\SpecialCharTok{{-}}\DecValTok{1}\NormalTok{))}
\NormalTok{\{}
\NormalTok{  AVT[}\StringTok{"Regression"}\NormalTok{, }\StringTok{"Df"}\NormalTok{]}\OtherTok{=}\NormalTok{AVT[}\StringTok{"Regression"}\NormalTok{, }\StringTok{"Df"}\NormalTok{]}\SpecialCharTok{+}\FunctionTok{as.double}\NormalTok{(chemical.lm.anova[i,}\StringTok{"Df"}\NormalTok{])}
\NormalTok{  AVT[}\StringTok{"Regression"}\NormalTok{, }\StringTok{"Sum Sq"}\NormalTok{]}\OtherTok{=}\NormalTok{AVT[}\StringTok{"Regression"}\NormalTok{, }\StringTok{"Sum Sq"}\NormalTok{]}\SpecialCharTok{+}\FunctionTok{as.double}\NormalTok{(chemical.lm.anova[i,}\StringTok{"Sum Sq"}\NormalTok{])}
\NormalTok{\}}
\NormalTok{AVT[}\StringTok{"Residuals"}\NormalTok{, }\StringTok{"Df"}\NormalTok{]}\OtherTok{=}\FunctionTok{as.double}\NormalTok{(chemical.lm.anova[}\StringTok{"Residuals"}\NormalTok{,}\StringTok{"Df"}\NormalTok{])}
\NormalTok{AVT[}\StringTok{"Residuals"}\NormalTok{, }\StringTok{"Sum Sq"}\NormalTok{]}\OtherTok{=}\FunctionTok{as.double}\NormalTok{(chemical.lm.anova[}\StringTok{"Residuals"}\NormalTok{,}\StringTok{"Sum Sq"}\NormalTok{])}
\NormalTok{AVT[}\StringTok{"Total"}\NormalTok{, }\StringTok{"Df"}\NormalTok{]}\OtherTok{=}\NormalTok{AVT[}\StringTok{"Regression"}\NormalTok{, }\StringTok{"Df"}\NormalTok{]}\SpecialCharTok{+}\NormalTok{AVT[}\StringTok{"Residuals"}\NormalTok{, }\StringTok{"Df"}\NormalTok{]}
\NormalTok{AVT[}\StringTok{"Total"}\NormalTok{, }\StringTok{"Sum Sq"}\NormalTok{]}\OtherTok{=}\NormalTok{AVT[}\StringTok{"Regression"}\NormalTok{, }\StringTok{"Sum Sq"}\NormalTok{]}\SpecialCharTok{+}\NormalTok{AVT[}\StringTok{"Residuals"}\NormalTok{, }\StringTok{"Sum Sq"}\NormalTok{]}
\NormalTok{AVT[}\StringTok{"Regression"}\NormalTok{, }\StringTok{"Mean Sq"}\NormalTok{] }\OtherTok{=}\NormalTok{ AVT[}\StringTok{"Regression"}\NormalTok{, }\StringTok{"Sum Sq"}\NormalTok{] }\SpecialCharTok{/}\NormalTok{ AVT[}\StringTok{"Regression"}\NormalTok{, }\StringTok{"Df"}\NormalTok{]}
\NormalTok{AVT[}\StringTok{"Residuals"}\NormalTok{, }\StringTok{"Mean Sq"}\NormalTok{] }\OtherTok{=}\NormalTok{ AVT[}\StringTok{"Residuals"}\NormalTok{, }\StringTok{"Sum Sq"}\NormalTok{] }\SpecialCharTok{/}\NormalTok{ AVT[}\StringTok{"Residuals"}\NormalTok{, }\StringTok{"Df"}\NormalTok{]}
\NormalTok{AVT[}\StringTok{"Regression"}\NormalTok{, }\StringTok{"F value"}\NormalTok{] }\OtherTok{=} \FunctionTok{round}\NormalTok{(AVT[}\StringTok{"Regression"}\NormalTok{, }\StringTok{"Mean Sq"}\NormalTok{] }\SpecialCharTok{/}\NormalTok{ AVT[}\StringTok{"Residuals"}\NormalTok{, }\StringTok{"Mean Sq"}\NormalTok{], }\DecValTok{1}\NormalTok{)}
\NormalTok{AVT[}\StringTok{"Regression"}\NormalTok{, }\StringTok{"Pr(\textgreater{}F)"}\NormalTok{] }\OtherTok{=} \FunctionTok{round}\NormalTok{(}\DecValTok{1}\SpecialCharTok{{-}}\FunctionTok{pf}\NormalTok{(AVT[}\StringTok{"Regression"}\NormalTok{, }\StringTok{"F value"}\NormalTok{], }
\NormalTok{                                         AVT[}\StringTok{"Regression"}\NormalTok{, }\StringTok{"Df"}\NormalTok{], }
\NormalTok{                                         AVT[}\StringTok{"Residuals"}\NormalTok{, }\StringTok{"Df"}\NormalTok{]), }\DecValTok{6}\NormalTok{)}
\FunctionTok{print}\NormalTok{(AVT)}
\end{Highlighting}
\end{Shaded}

\begin{verbatim}
##            Df    Sum Sq   Mean Sq F value   Pr(>F)
## Regression  2 1048.8114 524.40571    26.3 0.000174
## Residuals   9  179.4386  19.93762     0.0 0.000000
## Total      11 1228.2500   0.00000     0.0 0.000000
\end{verbatim}

\hypertarget{uxc794uxcc28-uxc0b0uxc810uxb3c4-uxb3c5uxb9bduxbcc0uxc218-uxc794uxcc28}{%
\subsection{6) 잔차 산점도 : (독립변수,
잔차)}\label{uxc794uxcc28-uxc0b0uxc810uxb3c4-uxb3c5uxb9bduxbcc0uxc218-uxc794uxcc28}}

\begin{Shaded}
\begin{Highlighting}[]
\CommentTok{\# 6) 잔차 산점도 : (독립변수, 잔차)}
\FunctionTok{par}\NormalTok{(}\AttributeTok{mfrow=}\FunctionTok{c}\NormalTok{(}\DecValTok{1}\NormalTok{,}\DecValTok{2}\NormalTok{), }\AttributeTok{pty=}\StringTok{"s"}\NormalTok{)}
\FunctionTok{plot}\NormalTok{(chemical}\SpecialCharTok{$}\NormalTok{speed, chemical.lm}\SpecialCharTok{$}\NormalTok{resid, }\AttributeTok{pch=}\DecValTok{19}\NormalTok{)}
\FunctionTok{abline}\NormalTok{(}\AttributeTok{h=}\DecValTok{0}\NormalTok{, }\AttributeTok{lty=}\DecValTok{2}\NormalTok{)}
\ControlFlowTok{for}\NormalTok{ (i }\ControlFlowTok{in} \DecValTok{1}\SpecialCharTok{:}\FunctionTok{length}\NormalTok{(chemical}\SpecialCharTok{$}\NormalTok{speed))}
\NormalTok{\{}
  \ControlFlowTok{if}\NormalTok{(}\FunctionTok{abs}\NormalTok{(chemical.lm}\SpecialCharTok{$}\NormalTok{resid[i]) }\SpecialCharTok{\textgreater{}} \FloatTok{3.7}\NormalTok{)}
\NormalTok{  \{}
    \FunctionTok{text}\NormalTok{(chemical}\SpecialCharTok{$}\NormalTok{speed[i]}\SpecialCharTok{+}\FloatTok{0.3}\NormalTok{, chemical.lm}\SpecialCharTok{$}\NormalTok{resid[i]}\SpecialCharTok{{-}}\FloatTok{0.4}\NormalTok{, }\FunctionTok{as.character}\NormalTok{(i))}
\NormalTok{  \}}
\NormalTok{\}}

\FunctionTok{plot}\NormalTok{(chemical}\SpecialCharTok{$}\NormalTok{temp, chemical.lm}\SpecialCharTok{$}\NormalTok{resid, }\AttributeTok{pch=}\DecValTok{19}\NormalTok{)}
\FunctionTok{abline}\NormalTok{(}\AttributeTok{h=}\DecValTok{0}\NormalTok{, }\AttributeTok{lty=}\DecValTok{2}\NormalTok{)}
\ControlFlowTok{for}\NormalTok{ (i }\ControlFlowTok{in} \DecValTok{1}\SpecialCharTok{:}\FunctionTok{length}\NormalTok{(chemical}\SpecialCharTok{$}\NormalTok{temp))}
\NormalTok{\{}
  \ControlFlowTok{if}\NormalTok{(}\FunctionTok{abs}\NormalTok{(chemical.lm}\SpecialCharTok{$}\NormalTok{resid[i]) }\SpecialCharTok{\textgreater{}} \FloatTok{3.7}\NormalTok{)}
\NormalTok{  \{}
    \FunctionTok{text}\NormalTok{(chemical}\SpecialCharTok{$}\NormalTok{temp[i]}\SpecialCharTok{+}\FloatTok{0.3}\NormalTok{, chemical.lm}\SpecialCharTok{$}\NormalTok{resid[i]}\SpecialCharTok{{-}}\FloatTok{0.3}\NormalTok{, }\FunctionTok{as.character}\NormalTok{(i)) }
\NormalTok{  \}}
\NormalTok{\}}
\end{Highlighting}
\end{Shaded}

\includegraphics{2장.중귀회귀모형-분석사례_files/figure-latex/unnamed-chunk-10-1.pdf}

\end{document}
