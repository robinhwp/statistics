% Options for packages loaded elsewhere
\PassOptionsToPackage{unicode}{hyperref}
\PassOptionsToPackage{hyphens}{url}
%
\documentclass[
]{article}
\usepackage{amsmath,amssymb}
\usepackage{lmodern}
\usepackage{iftex}
\ifPDFTeX
  \usepackage[T1]{fontenc}
  \usepackage[utf8]{inputenc}
  \usepackage{textcomp} % provide euro and other symbols
\else % if luatex or xetex
  \usepackage{unicode-math}
  \defaultfontfeatures{Scale=MatchLowercase}
  \defaultfontfeatures[\rmfamily]{Ligatures=TeX,Scale=1}
\fi
% Use upquote if available, for straight quotes in verbatim environments
\IfFileExists{upquote.sty}{\usepackage{upquote}}{}
\IfFileExists{microtype.sty}{% use microtype if available
  \usepackage[]{microtype}
  \UseMicrotypeSet[protrusion]{basicmath} % disable protrusion for tt fonts
}{}
\makeatletter
\@ifundefined{KOMAClassName}{% if non-KOMA class
  \IfFileExists{parskip.sty}{%
    \usepackage{parskip}
  }{% else
    \setlength{\parindent}{0pt}
    \setlength{\parskip}{6pt plus 2pt minus 1pt}}
}{% if KOMA class
  \KOMAoptions{parskip=half}}
\makeatother
\usepackage{xcolor}
\usepackage[margin=1in]{geometry}
\usepackage{color}
\usepackage{fancyvrb}
\newcommand{\VerbBar}{|}
\newcommand{\VERB}{\Verb[commandchars=\\\{\}]}
\DefineVerbatimEnvironment{Highlighting}{Verbatim}{commandchars=\\\{\}}
% Add ',fontsize=\small' for more characters per line
\usepackage{framed}
\definecolor{shadecolor}{RGB}{248,248,248}
\newenvironment{Shaded}{\begin{snugshade}}{\end{snugshade}}
\newcommand{\AlertTok}[1]{\textcolor[rgb]{0.94,0.16,0.16}{#1}}
\newcommand{\AnnotationTok}[1]{\textcolor[rgb]{0.56,0.35,0.01}{\textbf{\textit{#1}}}}
\newcommand{\AttributeTok}[1]{\textcolor[rgb]{0.77,0.63,0.00}{#1}}
\newcommand{\BaseNTok}[1]{\textcolor[rgb]{0.00,0.00,0.81}{#1}}
\newcommand{\BuiltInTok}[1]{#1}
\newcommand{\CharTok}[1]{\textcolor[rgb]{0.31,0.60,0.02}{#1}}
\newcommand{\CommentTok}[1]{\textcolor[rgb]{0.56,0.35,0.01}{\textit{#1}}}
\newcommand{\CommentVarTok}[1]{\textcolor[rgb]{0.56,0.35,0.01}{\textbf{\textit{#1}}}}
\newcommand{\ConstantTok}[1]{\textcolor[rgb]{0.00,0.00,0.00}{#1}}
\newcommand{\ControlFlowTok}[1]{\textcolor[rgb]{0.13,0.29,0.53}{\textbf{#1}}}
\newcommand{\DataTypeTok}[1]{\textcolor[rgb]{0.13,0.29,0.53}{#1}}
\newcommand{\DecValTok}[1]{\textcolor[rgb]{0.00,0.00,0.81}{#1}}
\newcommand{\DocumentationTok}[1]{\textcolor[rgb]{0.56,0.35,0.01}{\textbf{\textit{#1}}}}
\newcommand{\ErrorTok}[1]{\textcolor[rgb]{0.64,0.00,0.00}{\textbf{#1}}}
\newcommand{\ExtensionTok}[1]{#1}
\newcommand{\FloatTok}[1]{\textcolor[rgb]{0.00,0.00,0.81}{#1}}
\newcommand{\FunctionTok}[1]{\textcolor[rgb]{0.00,0.00,0.00}{#1}}
\newcommand{\ImportTok}[1]{#1}
\newcommand{\InformationTok}[1]{\textcolor[rgb]{0.56,0.35,0.01}{\textbf{\textit{#1}}}}
\newcommand{\KeywordTok}[1]{\textcolor[rgb]{0.13,0.29,0.53}{\textbf{#1}}}
\newcommand{\NormalTok}[1]{#1}
\newcommand{\OperatorTok}[1]{\textcolor[rgb]{0.81,0.36,0.00}{\textbf{#1}}}
\newcommand{\OtherTok}[1]{\textcolor[rgb]{0.56,0.35,0.01}{#1}}
\newcommand{\PreprocessorTok}[1]{\textcolor[rgb]{0.56,0.35,0.01}{\textit{#1}}}
\newcommand{\RegionMarkerTok}[1]{#1}
\newcommand{\SpecialCharTok}[1]{\textcolor[rgb]{0.00,0.00,0.00}{#1}}
\newcommand{\SpecialStringTok}[1]{\textcolor[rgb]{0.31,0.60,0.02}{#1}}
\newcommand{\StringTok}[1]{\textcolor[rgb]{0.31,0.60,0.02}{#1}}
\newcommand{\VariableTok}[1]{\textcolor[rgb]{0.00,0.00,0.00}{#1}}
\newcommand{\VerbatimStringTok}[1]{\textcolor[rgb]{0.31,0.60,0.02}{#1}}
\newcommand{\WarningTok}[1]{\textcolor[rgb]{0.56,0.35,0.01}{\textbf{\textit{#1}}}}
\usepackage{graphicx}
\makeatletter
\def\maxwidth{\ifdim\Gin@nat@width>\linewidth\linewidth\else\Gin@nat@width\fi}
\def\maxheight{\ifdim\Gin@nat@height>\textheight\textheight\else\Gin@nat@height\fi}
\makeatother
% Scale images if necessary, so that they will not overflow the page
% margins by default, and it is still possible to overwrite the defaults
% using explicit options in \includegraphics[width, height, ...]{}
\setkeys{Gin}{width=\maxwidth,height=\maxheight,keepaspectratio}
% Set default figure placement to htbp
\makeatletter
\def\fps@figure{htbp}
\makeatother
\setlength{\emergencystretch}{3em} % prevent overfull lines
\providecommand{\tightlist}{%
  \setlength{\itemsep}{0pt}\setlength{\parskip}{0pt}}
\setcounter{secnumdepth}{-\maxdimen} % remove section numbering
\ifLuaTeX
  \usepackage{selnolig}  % disable illegal ligatures
\fi
\IfFileExists{bookmark.sty}{\usepackage{bookmark}}{\usepackage{hyperref}}
\IfFileExists{xurl.sty}{\usepackage{xurl}}{} % add URL line breaks if available
\urlstyle{same} % disable monospaced font for URLs
\hypersetup{
  pdftitle={presentation},
  pdfauthor={robinhwp},
  hidelinks,
  pdfcreator={LaTeX via pandoc}}

\title{presentation}
\author{robinhwp}
\date{2023-03-22}

\begin{document}
\maketitle

\hypertarget{uxac15.-uxc81c2uxc7a5-uxc911uxd68cuxadc0uxbaa8uxd615-2}{%
\section{5강. 제2장
중회귀모형-2}\label{uxac15.-uxc81c2uxc7a5-uxc911uxd68cuxadc0uxbaa8uxd615-2}}

\hypertarget{uxd45cuxc900uxd654uxb41c-uxc911uxd68cuxadc0uxbd84uxc11d}{%
\subsection{1. 표준화된
중회귀분석}\label{uxd45cuxc900uxd654uxb41c-uxc911uxd68cuxadc0uxbd84uxc11d}}

\begin{itemize}
\tightlist
\item
  종속변수와 독립변수의 표준화 정규분포의 표준화변환 (통개학개론
  p103)에서 \(Z = { X - \mu \over \sigma }\) 식으로 변환하여 정규분포
  \(N(0, 1)\)을 따르게 되는 것처럼 변수의 표준화로
  회귀계수(\(\hat{\alpha}_i\))의 단위가 통일되어 비교가 가능해지게 된다.

  \begin{itemize}
  \item
    하지만 이 비교를 가지고 너무 단정적으로 결론을 내리는데는 주의하여야
    한다.
  \item
    \(\begin{aligned} Y_i^* = {Y_i - \bar{Y} \over \sqrt{S_{YY}}} \end{aligned}\),
    \(\begin{aligned} Z_{ij} = {X_{ij} - \bar{X_j} \over \sqrt{S_{jj}}} \end{aligned}\)

    \begin{itemize}
    \tightlist
    \item
      \(S_{YY}=\sum(Y_i-\bar{Y})^2\)
    \item
      \(S_{jj}=\sum_i (X_{ij} - \bar{X_j})^2\)
    \item
      \(\bar{X_j}\): \(j\)번째 독립변수의 평균값
    \end{itemize}
  \item
    \(\begin{aligned} \sum_{i} Z_{i\color{indianred}{j}} = 0, \sum_{i} (Z_{i\color{indianred}{j}})^2 = 1, (j=1,2,\cdots, k) \end{aligned}\)
  \item
    \(\sum Y_i^*=0, \sum(Y_i^*)^2=1\)
  \end{itemize}
\end{itemize}

\hypertarget{uxbcc0uxc218-uxd45cuxc900uxd654}{%
\subsubsection{변수 표준화}\label{uxbcc0uxc218-uxd45cuxc900uxd654}}

\begin{itemize}
\tightlist
\item
  표준화된 변수(standardized variables)의 중회귀모형

  \begin{itemize}
  \tightlist
  \item
    \(Y_i^*=\alpha_1 Z_{i1} + \alpha_2 Z_{i2} + \cdots + \alpha_k Z_{ik} + \varepsilon_i'\)
  \item
    표준화된 중회귀모형은 \(Y\)의 절편이 \(0\) 이 되는 특징이 있다
  \end{itemize}
\item
  표준화된 중회귀모형 추정하면

  \begin{itemize}
  \tightlist
  \item
    \(\hat{Y}_i^*=\hat{\alpha}_1 Z_{i1} + \hat{\alpha}_2 Z_{i2} + \cdots + \hat{\alpha}_k Z_{ik}\)
    \(\alpha_i\)의 절대값이 크면 클수록 독립변수 \(X_i\)가 \(Y_i\)에
    주는 영향이 크게 된다.
  \end{itemize}
\end{itemize}

\hypertarget{uxcd94uxc815uxacfc-uxac80uxc815}{%
\subsection{2. 추정과 검정}\label{uxcd94uxc815uxacfc-uxac80uxc815}}

\hypertarget{uxcd94uxc815uxb41c-uxd68cuxadc0uxacc4uxc218uxc758-uxbd84uxc0b0}{%
\subsubsection{추정된 회귀계수의
분산}\label{uxcd94uxc815uxb41c-uxd68cuxadc0uxacc4uxc218uxc758-uxbd84uxc0b0}}

\begin{itemize}
\item
  회귀계수벡터
  \(\boldsymbol{\beta}'=(\beta_0,\beta_1,\beta_2,\cdots,\beta_k)\)의
  \(\boldsymbol{\beta}\)의 추정량 \(\boldsymbol{\hat{\beta}}\)\\
  \(\color{blue}{E(\boldsymbol{\hat{\beta}}) = \boldsymbol{\beta}}\)

  \(\begin{aligned} E(\boldsymbol{\hat{\beta}}) &= E[(\boldsymbol{X'X})^{-1}\boldsymbol{X'Y}] \\&=(\boldsymbol{X'X})^{-1}\boldsymbol{X'}E(\boldsymbol{Y}) \\&= (\boldsymbol{X'X})^{-1}\boldsymbol{X'X\beta} \\&=\boldsymbol{\beta} \end{aligned}\)

  \(\boldsymbol{\hat{\beta}}\)는 \(\boldsymbol{\beta}\)의 불편추정량
\item
  \(\boldsymbol{\hat{\beta}}\)의 분산-공분산 행렬은\\
  \(\color{blue} {Var(\boldsymbol{\hat{\beta}})= (\boldsymbol{X'X})^{-1}\sigma^2}\)\\
  \(\begin{aligned} Var(\boldsymbol{\hat{\beta}}) &=Var[(\boldsymbol{X'X})^{-1}\boldsymbol{X'Y}] \\&=(\boldsymbol{X'X})^{-1}\boldsymbol{X'}Var[\boldsymbol{Y}][(\boldsymbol{X'X})^{-1}\boldsymbol{X'}]' \\&=(\boldsymbol{X'X})^{-1}\boldsymbol{X'}[\boldsymbol{I}\sigma^2]\boldsymbol{X}(\boldsymbol{X'X})^{-1} \\&=(\boldsymbol{X'X})^{-1}\boldsymbol{X'}\boldsymbol{X}(\boldsymbol{X'X})^{-1}\sigma^2 \\&=(\boldsymbol{X'X})^{-1}\sigma^2 \end{aligned}\)
\end{itemize}

그런데 \(Var(\boldsymbol{\hat{\beta}})\)는 벡터
\(\boldsymbol{\hat{\beta}}\)의 구성원들 간의 분산과 공분산을 나타내는
행렬로서 \((\boldsymbol{X’X})^{-1}\)행렬의 구성원을\\
\(c_{ij}(i,j=0,1,2,\cdots, k)\)라 하면 다음과 같이 쓸수 있다.

\begin{itemize}
\tightlist
\item
  \(Var(\hat{\beta}_i)=c_{ii}\sigma^2, (i = 0, 1, 2, \cdots, k)\)
\item
  \(Cov(\hat{\beta}_i, \hat{\beta}_j)=c_{ij}\sigma^2, (i \ne j)\)
\end{itemize}

만약 우리가 특별히 관심 있는 독립변수 \(X_i\)라면 \(b_i\)의 분산이
작게되는 회귀방정식이 요구될 것이며,\\
\(c_{ii}\)의 값이 크지 않도록 계획행렬(design matrix)
\(\boldsymbol{X}\)를 설계해 줄 필요가 있을 것이다.\\
\(\boldsymbol{X}\) 행렬의 설계에 따라서
\(Var(\boldsymbol{\hat{\beta}})\)가 크게 달라지기 때문이다.

\hypertarget{eyuxc758-uxad6cuxac04uxcd94uxc815}{%
\subsubsection{\texorpdfstring{\(E(Y)\)의
구간추정}{E(Y)의 구간추정}}\label{eyuxc758-uxad6cuxac04uxcd94uxc815}}

\begin{itemize}
\tightlist
\item
  독립변수들의 임의의 값 (\(x_1, x_2, \cdots, x_k\))에서
  \(E(\boldsymbol{Y})\)의 구간추정
\end{itemize}

\(\hat{Y} \pm t(n-k-1;\alpha/2)\sqrt{\boldsymbol{x'}(\boldsymbol{X'X})^{-1}\boldsymbol{x}\cdot MSE}\)

\(\begin{aligned}\boldsymbol{\hat{Y}}&=\hat{\beta}_0 + \hat{\beta}_1 x_1 + \hat{\beta}_2 x_2 + \cdots + \hat{\beta}_k x_k \\&= \begin{pmatrix} 1, x_1,x_2,\cdots,x_k \end{pmatrix} \begin{pmatrix} \hat{\beta}_0 \\ \hat{\beta}_1 \\ \hat{\beta}_2 \\ \vdots \\ \hat{\beta}_k \end{pmatrix} \\&= \boldsymbol{x'\hat{\beta}}\end{aligned}\)

⇒
\(\begin{aligned} Var(\hat{Y})&=Var(\boldsymbol{x'\hat{\beta}})=\boldsymbol{x'}Var(\boldsymbol{\hat{\beta}})\boldsymbol{x} \\&=\boldsymbol{x}'(\boldsymbol{X'X})^{-1}\boldsymbol{x}\sigma^2 \end{aligned}\)

⇒ \(E(Y)\)의 \(100(1-\alpha)\)\% 신뢰구간\\
\(\color{blue}{\hat{Y} \pm t(n-k-1;\alpha/2)\sqrt{\boldsymbol{x'}(\boldsymbol{X'X})^{-1}\boldsymbol{x}\cdot MSE}}\)

\hypertarget{uxd68cuxadc0uxacc4uxc218-beta_iuxc758-uxac00uxc124-uxbc0f-uxac80uxc815uxd1b5uxacc4uxb7c9}{%
\subsubsection{\texorpdfstring{회귀계수 \(\beta_i\)의 가설 및
검정통계량}{회귀계수 \textbackslash beta\_i의 가설 및 검정통계량}}\label{uxd68cuxadc0uxacc4uxc218-beta_iuxc758-uxac00uxc124-uxbc0f-uxac80uxc815uxd1b5uxacc4uxb7c9}}

\begin{itemize}
\tightlist
\item
  가설

  \begin{itemize}
  \tightlist
  \item
    \(H_0 : \beta_i = \beta_{i0}\)
  \item
    \(H_1 : \beta_i \ne \beta_{i0}\)
  \end{itemize}
\item
  검정통계량

  \begin{itemize}
  \tightlist
  \item
    \(\begin{aligned} t_0={b_i - \beta_{i0} \over \sqrt{c_{ii}\cdot MSE}} \end{aligned}\),
    \(c_{ii}\) 는
    \(Var(\boldsymbol{\hat{\beta}})= (\boldsymbol{X'X})^{-1}\sigma^2\)의
    대각선 값
  \end{itemize}
\end{itemize}

\hypertarget{uxc77cuxbc18uxc801-uxbaa8uxd615uxbe44uxad50}{%
\subsubsection{일반적
모형비교}\label{uxc77cuxbc18uxc801-uxbaa8uxd615uxbe44uxad50}}

\begin{itemize}
\item
  두 모형의 비교는 \texttt{완전모형}과 \texttt{축소모형}의 잔차제곱합의
  차이를 이용함 두 모형을 비교하기위한 검정통계량은
  \(\begin{aligned} F_0 = {MSR \over MSE} \end{aligned}\) 을 사용한다.

  \texttt{완전모형} : 데이터에 잘 적합되리라고 고려되는 모형 (
  \(Y_i = \beta_0 + \beta_1 X_i + \varepsilon_i\) )

  \begin{itemize}
  \tightlist
  \item
    완전모형의 경우의 잔차제곱합

    \begin{itemize}
    \tightlist
    \item
      \(\hat{Y}=b_0 + b_1X\)
    \item
      \(SSE(F)=\sum [Y_i-(b_0 + b_1 X_i)]^2 = \sum (Y_i - \hat{Y}_i)^2 = SSE\)
    \end{itemize}
  \end{itemize}

  \texttt{축소모형} : 귀무가설 \(H_0 : \beta_1 = 0\) 의 가정하에서의
  모형 ( \(Y_i = \beta_0 + \varepsilon_i\) )

  \begin{itemize}
  \item
    축소모형의 경우 잔차제곱합 ( \(\beta_1 = 0\) 이라고 가정 )

    \begin{itemize}
    \tightlist
    \item
      \(\hat{Y}=b_0\)
    \item
      \(SSE(R)=\sum (Y_i-b_0)^2 = \sum (Y_i - \bar{Y})^2 = SST\)
    \end{itemize}
  \item
    두 모형을 비교하기위한 검정통계량

    \begin{itemize}
    \tightlist
    \item
      \(\begin{aligned} F_0 = {MSR \over MSE} \end{aligned}\)\\
      \(\begin{aligned} F_0 = {{[SSE(R)-SSE(F)] \over df_R - df_F} \over {SSE(F) \\over df_F}} &\Rightarrow {{[SST-SSE] \over (n-1) - (n-2)} \over {SSE \\over n-2}} \\&= {{SSR} \over {SSE \\over n-2}} = {MSR \over MSE} \end{aligned}\)
    \end{itemize}
  \end{itemize}
\end{itemize}

\hypertarget{uxbcc0uxc218uxcd94uxac00}{%
\subsection{3. 변수추가}\label{uxbcc0uxc218uxcd94uxac00}}

\hypertarget{uxcd94uxac00uxc81cuxacf1uxd569}{%
\subsubsection{추가제곱합}\label{uxcd94uxac00uxc81cuxacf1uxd569}}

\begin{itemize}
\tightlist
\item
  중회귀모형을 적합하는데 있어서 어떤 특정한 변수를 회귀모형에
  포함시키는 것이 바람직한가를 결정하고 싶은 경우

  \begin{itemize}
  \tightlist
  \item
    이 변수를 포함시키지 않고 구한 회귀제곱합보다 이 변수를 포함시키고
    구한 회귀제곱합(regression sum of squares )이 추가적으로 어느 정도
    커졌는가를 검토. 이와 같은 경우에 추가적으로 증가된 제곱합을
    추가제곱합(extra sum of squares)이라고 함.
  \item
    추가제곱합은 새로운 변수가 모형에 추가될 때의 회귀제곱합의 증가분을
    나타내는 것으로서 이 값이 작을수록 회귀에 대한 기여도가 떨어진다는
    것을 의미.
  \end{itemize}
\end{itemize}

\hypertarget{uxcd94uxac00uxbcc0uxc218uxadf8uxb9bcadded-variable-plot}{%
\subsubsection{추가변수그림(added variable
plot)}\label{uxcd94uxac00uxbcc0uxc218uxadf8uxb9bcadded-variable-plot}}

\begin{itemize}
\item
  중회귀모형에서 새로운 변수선택은 기존의 모형이 설명하지 못하는 부분을
  새로운 변수가 들어옴으로써 추가설명력이 얼마나 유의한 가에 따라 결정

  \begin{itemize}
  \tightlist
  \item
    새로운 변수의 효과를 그래프로 표현할 수 있는데, 이러한 그래프 중의
    하나가 추가변수그림(added variable plot)임. 이를 편회귀그림(partial
    regression plot)이라고도 함.
  \end{itemize}
\item
  추가변수 그림 그리는 절차

  \begin{itemize}
  \tightlist
  \item
    독립변수가 2개인 회귀모형에서 변수 \(X_2\)의 추가변수그림을 그리는
    절차

    \begin{itemize}
    \tightlist
    \item
      \(Y = \beta_0 + \beta_1 X_1 + \beta_2 X_2 + \varepsilon\)
    \end{itemize}
  \end{itemize}

  \begin{enumerate}
  \def\labelenumi{\arabic{enumi}.}
  \tightlist
  \item
    \(Y\)를 \(X_1\)으로 회귀한 후 얻어지는 잔차, \(e(Y|X_1)\)을 구한다.
  \item
    \(X_2\)를 \(X_1\)으로 회귀한 후 얻어지는 잔차, \(e(X_2|X_1)\)을
    구한다.
  \item
    앞에서 구한 두 잔차에서 \(x\)-축을 \(e(X_2|X_1)\), \(y\)-축을
    \(e(Y|X_1)\)으로 한 산점도를 추가변수 그림이라 함.
  \end{enumerate}

  → 추가변수그림이 선형관계가 있으면 변수 \(X_2\)는 추가적인 설명력이
  있다고 판단.

  위의 방식대로 강의에 있는 추가변수그림 첫번째 그래프를 그려보는 테스트
  코드.

\begin{Shaded}
\begin{Highlighting}[]
\CommentTok{\# 실습코드 : 잔차 e(X1|X2,X3,X4)와 잔차 e(Y|X2,X3,X4)의 산점도}
\NormalTok{health }\OtherTok{=} \FunctionTok{read.table}\NormalTok{(}\StringTok{"./data/health.txt"}\NormalTok{, }\AttributeTok{header=}\NormalTok{T)}
\CommentTok{\# head(health,3)}
\NormalTok{h4.lm }\OtherTok{=} \FunctionTok{lm}\NormalTok{(Y }\SpecialCharTok{\textasciitilde{}}\NormalTok{ X1}\SpecialCharTok{+}\NormalTok{X2}\SpecialCharTok{+}\NormalTok{X3}\SpecialCharTok{+}\NormalTok{X4, }\AttributeTok{data=}\NormalTok{health)}
\NormalTok{h4.lm.y1 }\OtherTok{=} \FunctionTok{lm}\NormalTok{(Y }\SpecialCharTok{\textasciitilde{}}\NormalTok{ X2}\SpecialCharTok{+}\NormalTok{X3}\SpecialCharTok{+}\NormalTok{X4, }\AttributeTok{data=}\NormalTok{health)}
\NormalTok{h4.lm.x1 }\OtherTok{=} \FunctionTok{lm}\NormalTok{(X1 }\SpecialCharTok{\textasciitilde{}}\NormalTok{ X2}\SpecialCharTok{+}\NormalTok{X3}\SpecialCharTok{+}\NormalTok{X4, }\AttributeTok{data=}\NormalTok{health)}
\NormalTok{h4.lm11}\OtherTok{=}\FunctionTok{lm}\NormalTok{(}\FunctionTok{resid}\NormalTok{(h4.lm.y1)}\SpecialCharTok{\textasciitilde{}}\FunctionTok{resid}\NormalTok{(h4.lm.x1))}
\FunctionTok{plot}\NormalTok{(}\FunctionTok{resid}\NormalTok{(h4.lm.x1),}\FunctionTok{resid}\NormalTok{(h4.lm.y1), }\AttributeTok{xlab=}\StringTok{"X1 | others"}\NormalTok{, }\AttributeTok{ylab =} \StringTok{"Y | others"}\NormalTok{)}
\FunctionTok{abline}\NormalTok{(h4.lm11, }\AttributeTok{col=}\StringTok{"red"}\NormalTok{, }\AttributeTok{lwd=}\DecValTok{2}\NormalTok{)}
\end{Highlighting}
\end{Shaded}
\end{itemize}

\hypertarget{uxc794uxcc28uxac80uxd1a0-uxbc0f-uxbd84uxc11d-uxc0acuxb840}{%
\subsection{4. 잔차검토 및 분석
사례}\label{uxc794uxcc28uxac80uxd1a0-uxbc0f-uxbd84uxc11d-uxc0acuxb840}}

\hypertarget{uxc794uxcc28uxc758-uxac80uxd1a0}{%
\subsubsection{잔차의 검토}\label{uxc794uxcc28uxc758-uxac80uxd1a0}}

\includegraphics{5강 제2장 중회귀모형-2 8bcef66ab07b4fd6b163768d59a06c84/Untitled.png}

\textbf{잔차(}\(e_i\))를 \(\hat{Y}_i\)에 대한 산점도를 그렸을때,

ⓐ 가정에 아무런 모순이 없는것으로 판정된다.

ⓑ 분산이 일정하지 않으며, 가중회귀(weighted regression)를 쓰거나 또는
\(Y_i\)를 변환시켜 회귀분석함이 바람직하다.

ⓒ 절편이 필요한 모형인데 절편을 사용하지 않았을 경우에 생길 수 있는
형태이다.

ⓓ 모형이 타당하지 않다. 추가적으로 독립변수의 제곱항이 필요하다. 또는
\(Y_i\)의 적절한 변환이 필요하다.

\textbf{잔차(}\(e_i\))를 독립변수 \(X_{ij}, j= 1,2,...,k\) 에 대한
산점도를 그렸을때

ⓐ 가정에 아무런 모순이 없는 경우로 사용된 중회귀모형이 적절한다.

ⓑ 분산이 일정하지 않으며, 가중회귀(weighted regression)를 쓰거나 또는
\(Y_i\)를 변환시켜 회귀분석함이 바람직하다.

ⓒ 중회귀분석 과정에서 계산상에 착오가 있는 경우이다. \(X_{ij}\)의
선형효과가 적절히 취급되지 않았다, 즉 \(X_{ij}\)항을 빠뜨려서 계산했거나
계산상에 실수가 있다.

ⓓ 필요한 독립변수의 항들이 포함되어 있지 않다. 특히 필요한 제곱항이
빠졌을 경우에 발생한다.

\hypertarget{section}{%
\subsubsection{}\label{section}}

\end{document}
